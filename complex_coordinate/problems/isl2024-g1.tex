\begin{bprb}[ISL2024-G1]
円に内接する四角形$ABCD$が$AC<BD<AD$および$\angle DBA<90^\circ$をみたしている.
$D$を通り$AB$に平行な直線上に点$E$をとったところ, $E$と$C$は直線$AD$に関して反対側にあり, $AC=DE$が成り立った.
$A$を通り$CD$に平行な直線上に点$F$をとったところ, $F$と$B$は直線$AD$に関して反対側にあり, $BD=AF$が成り立った.
$BC$の垂直二等分線と$EF$の垂直二等分線とは, 四角形$ABCD$の外接円上で交わることを示せ.
\end{bprb}
\begin{ifsol*}
$\lvert a\rvert=\lvert b\rvert=\lvert c\rvert=\lvert d\rvert=1$となるような座標で考える.
四角形$ABCD$の外接円を$\Omega$とおく.
$\sqrt{bc}$が点$A$を含まない弧$BC$上にあるように$\sqrt{bc}$の符号を定める.

$e-d$は$a-c$を$-\frac 12\angle BAC$だけ回転したものであるため, $e-d=(a-c)\sqrt{\frac bc}$が成り立つ.
同様に, $f-a$は$d-b$を$\frac 12\angle BDC$だけ回転したものであるため, $f-a=(d-b)\sqrt{\frac cb}$が成り立つ.

$-\sqrt{bc}$は$BC$の垂直二等分線上にあるので$-\sqrt{bc}$が$EF$の垂直二等分線上にあることを示せば十分である.
\[e+\sqrt{bc}=(a-c)\sqrt\frac bc+d+\sqrt{bc}=a\sqrt\frac bc+d\]
および
\[f+\sqrt{bc}=(d-b)\sqrt\frac cb+a+\sqrt{bc}=d\sqrt\frac cb+a\]
が成り立つので$\lvert e+\sqrt{bc}\rvert=\lvert f+\sqrt{bc}\rvert$であり, $-\sqrt{bc}$が$EF$の垂直二等分線上にあることが示された.
\end{ifsol*}