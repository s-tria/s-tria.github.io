\begin{bprb}[ISL2024-G5]
三角形$ABC$とその内心$I$とがあり, 三角形$BIC$の外接円を$\Omega$とする.
$K$は辺$BC$上(端点を除く)の点で, $\angle BAK<\angle KAC$をみたす.
$\angle BKA$の二等分線は$\Omega$と$2$点$W$, $X$で交わり, $\angle CKA$の二等分線は$\Omega$と$2$点$Y$, $Z$で交わった.
ただし, $W$および$Y$は直線$BC$に関して$A$と同じ側にあるとする.
このとき, $\angle WAY=\angle ZAX$を示せ.
\end{bprb}
\begin{ifsol*}
$\lvert b\rvert=\lvert c\rvert=\lvert i\rvert=1$となるような座標で考える.
定理\ref{thm:a-m}により$a=\frac{bc+i^2}{b+c}$である.
$\angle WAY=\measuredangle WAY$および$\angle ZAX=\measuredangle ZAX$が成り立つので, 示すべきことは$(w-a)(x-a)\sim(y-a)(z-a)$である.

直線$AK$と$\Omega$との交点を$P$, $Q$とする.
$A$は直線$PQ$上にあるので$l(p,q)$, $l(i,-i)$, $l(b,i^2/c)$は共線である.
したがって, 系\ref{cor:concurrency2}により
\begin{align*}
0
&=\begin{vmatrix}
pq&p+q&1\\
-i^2&0&1\\
\frac{bi^2}c&b+\frac{i^2}c&1
\end{vmatrix}\\
&=\frac 1c\begin{vmatrix}
pq&p+q&1\\
-i^2&0&1\\
bi^2&bc+i^2&c
\end{vmatrix}\\
&=\frac 1c\begin{vmatrix}
pq+i^2&p+q&0\\
-i^2&0&1\\
(b+c)i^2&bc+i^2&0
\end{vmatrix}\\
&=\frac{(p+q)(b+c)i^2-(pq+i^2)(bc+i^2)}{c}
\end{align*}
である.
これにより,
\[p+q=\frac{(pq+i^2)(bc+i^2)}{(b+c)i^2}\]
である.

直線$WX$は$\angle BKA$の二等分線なので$(wx)^2=bcpq$である.
$wx=\sqrt{bcpq}$, $yz=-\sqrt{bcpq}$となるように$\sqrt{bcpq}$の符号を定める.
直線$BC$, $PQ$, $WX$, $YZ$は共点なので
\[\begin{vmatrix}wx&w+x&1\\pq&p+q&1\\bc&b+c&1\end{vmatrix}=\begin{vmatrix}yz&y+z&1\\pq&p+q&1\\bc&b+c&1\end{vmatrix}=0\]
が成り立つ.
これにより
\[w+x=\frac{wx(p+q-b-c)+pq(b+c)-bc(p+q)}{pq-bc}\]
が成り立つ.

以上をあわせると
\begin{align*}
(w-a)(x-a)
&=wx-(w+x)\frac{bc+i^2}{b+c}+\biggl(\frac{bc+i^2}{b+c}\biggr)^2\\
&=wx-\frac{bc+i^2}{b+c}\frac{wx(p+q-b-c)+pq(b+c)-bc(p+q)}{pq-bc}+\biggl(\frac{bc+i^2}{b+c}\biggr)^2\\
&=wx-\frac{bc+i^2}{b+c}\frac{wx\Bigl(\frac{(pq+i^2)(bc+i^2)}{(b+c)i^2}-b-c\Bigr)+pq(b+c)-bc\frac{(pq+i^2)(bc+i^2)}{(b+c)i^2}}{pq-bc}+\biggl(\frac{bc+i^2}{b+c}\biggr)^2\\
&=\frac 1{(b+c)^2i^2(pq-bc)}\biggl(\Bigl((b+c)^2i^2(pq-bc)-(bc+i^2)\bigl((pq+i^2)(bc+i^2)-(b+c)^2i^2\bigr)\Bigr)wx\\
&\qquad\qquad-(bc+i^2)\bigl(pq(b+c)^2i^2-bc(pq+i^2)(bc+i^2)\bigr)+(bc+i^2)^2i^2(pq-bc)\biggr)\\
&=\frac 1{(b+c)^2i^2(pq-bc)}\biggl(\Bigl(-i^6+(b^2+c^2-pq)i^4+(b^2pq+c^2pq-b^2c^2)i^2-b^2c^2pq\Bigr)wx\\
&\qquad\qquad+\Bigl(i^6-(b^2-bc+c^2)i^4-(b^2-bc+c^2)bci^2+b^3c^3\Bigr)pq\biggr)
\end{align*}
と計算できる.
\begin{align*}
-i^6+(b^2+c^2-pq)i^4+(b^2pq+c^2pq-b^2c^2)i^2-b^2c^2pq&\sim i^3bc\sqrt{pq},\\
i^6-(b^2-bc+c^2)i^4-(b^2-bc+c^2)bci^2+b^3c^3&\sim i^3bc\sqrt{bc}
\end{align*}
から
\[(w-a)(x-a)\sim\frac{i^3bcpq\sqrt{bc}}{(b+c)^2i^2(pq-bc)}\]
がわかる.
$(y-a)(z-a)$は$(w-a)(x-a)$において$\sqrt{bcpq}$を$-\sqrt{bcpq}$に変えたものなので
\[(y-a)(z-a)\sim\frac{i^3bcpq\sqrt{bc}}{(b+c)^2i^2(pq-bc)}\]
もわかる.
したがって,
\[(w-a)(x-a)\sim(y-a)(z-a)\]
が示された.
\end{ifsol*}