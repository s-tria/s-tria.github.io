\begin{bprb}
三角形$ABC$の内心を$I$, 外心を$O$, Feuerbach点を$F_e$とする.
辺$BC$の中点を$M$, 三角形$ABC$の内接円が辺$BC$と接する点を$D$とするとき, 三角形$AIO$と三角形$F_eDM$とは相似であることを示せ.
\end{bprb}
\begin{ifsol*}
$i=0$, $\lvert d\rvert=1$となるような座標で考える.
定理\ref{thm:circumcenter-i}, 定理\ref{thm:feuerbach}により
\begin{align*}
a&=\frac{2ef}{e+f},\\
i&=0,\\
o&=\frac{2def(d+e+f)}{(d+e)(e+f)(f+d)},\\
f_e&=\frac{de+ef+fd}{d+e+f},\\
d&=d,\\
m&=\frac{d(de+df+2ef)}{(d+e)(d+f)}
\end{align*}
である.
示すべき式は
\[\begin{vmatrix}
\frac{2ef}{e+f}&\frac{de+ef+fd}{d+e+f}&1\\
0&d&1\\
\frac{2def(d+e+f)}{(d+e)(e+f)(f+d)}&\frac{d(de+df+2ef)}{(d+e)(d+f)}&1
\end{vmatrix}=0\]
である.

\begin{align*}
&\phantom{={}}\begin{vmatrix}
\frac{2ef}{e+f}&\frac{de+ef+fd}{d+e+f}&1\\
0&d&1\\
\frac{2def(d+e+f)}{(d+e)(e+f)(f+d)}&\frac{d(de+df+2ef)}{(d+e)(d+f)}&1
\end{vmatrix}\\
&=\frac{2ef}{(d+e)(e+f)(f+d)}\begin{vmatrix}
1&\frac{de+ef+fd}{d+e+f}&1\\
0&d&1\\
d(d+e+f)&d(de+df+2ef)&(d+e)(d+f)
\end{vmatrix}\\
&=\frac{2ef}{(d+e)(e+f)(f+d)(d+e+f)}\begin{vmatrix}
d+e+f&de+ef+fd&d+e+f\\
0&d&1\\
d(d+e+f)&d(de+df+2ef)&(d+e)(d+f)
\end{vmatrix}\\
&=\frac{2ef}{(d+e)(e+f)(f+d)(d+e+f)}\begin{vmatrix}
d+e+f&de+ef+fd&d+e+f\\
0&d&1\\
0&def&ef
\end{vmatrix}\\
&=0
\end{align*}
により示された.
\end{ifsol*}