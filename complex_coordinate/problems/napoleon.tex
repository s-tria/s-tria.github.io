\begin{bprb}[Napoleonの定理]
三角形$ABC$において, $BPC$, $CQA$, $ARB$が正三角形となるように$3$点$P$, $Q$, $R$をとる.
ただし, $(B,P,C)$, $(C,Q,A)$, $(A,R,B)$はいずれもこの順に反時計回りに並ぶとする.
三角形$BPC$, $CQA$, $ARB$の重心をそれぞれ$L$, $M$, $N$とするとき, 三角形$LMN$は正三角形であることを示せ.
\end{bprb}
\begin{ifsol*}
$\omega=\frac{-1+\sqrt{-3}}{2}$とおく.
三角形$BPC$が正三角形なので
\[p=-\omega c-\omega^2b\]
が成り立つ.
同様に
\[q=-\omega a-\omega^2c,\quad p=-\omega b-\omega^2a\]
も成り立つ.
示したい式は
\[l+\omega m+\omega^2n=0\]
であり,
\begin{align*}
l+\omega m+\omega^2n
&=\frac 13\bigl(b+c+p+\omega(c+a+q)+\omega^2(a+b+r)\bigr)\\
&=\frac 13\bigl(b+c-\omega c-\omega^2b+\omega(c+a-\omega a-\omega^2c)+\omega^2(a+b-\omega b-\omega^2a)\bigr)\\
&=0
\end{align*}
により成り立つ.
\end{ifsol*}