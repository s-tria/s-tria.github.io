本章にはここまでで紹介してきた公式を用いて解くことができる問題を多数掲載した.
これらの問題は概ね難易度の順になるように並べている.
全ての問題に解答を付しているので, 解くことも解答を読むことも勉強になるであろう.
\subsection{複素座標で実際の問題を解くには}
矛盾しているように聞こえるかもしれないが, 複素座標で問題を解くことにおいて初等幾何の力は大いに役に立つ.
実際, 第3章で行ったように, 点の座標を計算する前にその点のより計算しやすい特徴づけを与えることは重要である.
また, 初等的考察によって, 示したい命題を複素座標で示しやすい命題に言い換えることも重要である.
例えば, 根心の構図を用いて共円の条件を共線に帰着させることはよく行われる.
複素座標は単位円上の$2$点を結ぶ直線, 角度の条件を扱うことには長けているが, 単位円上にない点や辺の長さの条件に対しては弱い.

ここまでは座標計算の補助として初等的考察を行うことについて述べたが, 初等的考察の補助として座標計算を行うことができるようになると解ける問題の幅はさらに広がるであろう.

また, 座標計算を行う際には計算ミスが天敵である.
月並みなアドバイスではあるが, あとで検算をするときに見返してわかるように式変形を書くのは大事である.
そのほか, $\alpha+\beta$と$\bar\alpha+\bar\beta$を両方計算してそれらが共軛になっていることを確かめるなど, ダブルチェックを行うことも計算ミスを減らすことに一役買う.

もしあなたが数学オリンピックの受験生であるなら, 全ての幾何の問題を座標計算で解けると考えることは得策ではない.
本章には多数の問題を掲載したが, 掲載しなかった問題も多数あることを念頭に置くべきである.
座標計算で解ける問題であっても試験時間のうちに解ききれるとは限らない.
%
% \subsection{コンテスト名}
% コンテスト名などの略称についての説明は以下の通りである.
% \par\textbf{Balkan MO}: Balkan Mathematical Olympiad, バルカン数学オリンピック. 1問10点の40点満点.
% \par\textbf{EGMO}: European Girls' Mathematical Olympiad, ヨーロッパ女子数学オリンピック. 1問7点の42点満点.
% \par\textbf{IMO}: International Mathematical Olympiad, 国際数学オリンピック. 1問7点の42点満点.
% \par\textbf{ISL}: IMO shortlist. IMOの候補問題であり, この中からIMOの問題が選ばれるが, IMO開催から1年間はリストは公開されない. TSTにISLの問題を使う国も多い.
% \par\textbf{Japan TST}: Japan Team Selection Test. IMO日本代表の選抜試験. 1問7点の84点満点.
% \par\textbf{Vietnum TST}: Vietnum Team Selction Test.
\newcommand{\plabel}[1]{\label{#1}}
\showsolutionsfalse
\subsection{有名定理}
\begin{bprb}[Napoleonの定理]
三角形$ABC$において, $BPC$, $CQA$, $ARB$が正三角形となるように$3$点$P$, $Q$, $R$をとる.
ただし, $(B,P,C)$, $(C,Q,A)$, $(A,R,B)$はいずれもこの順に反時計回りに並ぶとする.
三角形$BPC$, $CQA$, $ARB$の重心をそれぞれ$L$, $M$, $N$とするとき, 三角形$LMN$は正三角形であることを示せ.
\end{bprb}
\begin{ifsol*}
$\omega=\frac{-1+\sqrt{-3}}{2}$とおく.
三角形$BPC$が正三角形なので
\[p=-\omega c-\omega^2b\]
が成り立つ.
同様に
\[q=-\omega a-\omega^2c,\quad p=-\omega b-\omega^2a\]
も成り立つ.
示したい式は
\[l+\omega m+\omega^2n=0\]
であり,
\begin{align*}
l+\omega m+\omega^2n
&=\frac 13\bigl(b+c+p+\omega(c+a+q)+\omega^2(a+b+r)\bigr)\\
&=\frac 13\bigl(b+c-\omega c-\omega^2b+\omega(c+a-\omega a-\omega^2c)+\omega^2(a+b-\omega b-\omega^2a)\bigr)\\
&=0
\end{align*}
により成り立つ.
\end{ifsol*}
\begin{bprb}[Morleyの定理]
三角形$ABC$において, 角$A$の内角の三等分線を$l_{ab}$, $l_{ac}$とおく.
ただし, 辺$AB$, $l_{ab}$, $l_{ac}$, 辺$AC$はこの順に並ぶとする.
$l_{ba}$, $l_{bc}$, $l_{ca}$, $l_{cb}$も同様に定める.
$l_{bc}$と$l_{cb}$との交点を$P$, $l_{ca}$と$l_{ac}$との交点を$Q$, $l_{ab}$と$l_{ba}$との交点を$R$としたとき, 三角形$PQR$は正三角形であることを示せ.
\end{bprb}
\begin{ifsol*}
$a=\alpha^3$, $b=\beta^3$, $c=\gamma^3$とおく.
ただし, $0\leq\arg\alpha\leq\arg\beta\leq\arg\gamma<\frac{\pi}{3}$とする.
$l_{ab}=l(\alpha^3,\beta^2\gamma)$, $l_{ac}=l(\alpha^3,\beta\gamma^2)$, $l_{bc}=l(\beta^3,\omega\gamma^2\alpha)$, $l_{ba}=l(\beta^3,\omega^2\gamma\beta^2)$, $l_{ca}=(\gamma^3,\alpha^2\beta)$, $l_{cb}=l(\gamma^3,\alpha\beta^2)$であることを用いる.
ただし, $\omega=e^{\frac{2\sqrt{-1}\pi}{3}}$である.
これにより, $P$の座標は
\begin{align*}
p
&=\frac{\beta^3\omega\gamma^2\alpha(\gamma^3+\alpha\beta^2)-\gamma^3\alpha\beta^2(\beta^3+\omega\gamma^2\alpha)}{\beta^3\omega\gamma^2\alpha-\gamma^3\alpha\beta^2}\\
&=\frac{\omega\beta(\gamma^3+\alpha\beta^2)-\gamma(\beta^3+\omega\gamma^2\alpha)}{\omega\beta-\gamma}\\
&=-\omega\beta\gamma^2-\omega^2\beta^2\gamma+\alpha\beta^2+\omega^2\alpha\beta\gamma+\omega\alpha\gamma^2,
\end{align*}
$Q$の座標は
\begin{align*}
q
&=\frac{\gamma^3\alpha^2\beta(\alpha^3+\beta\gamma^2)-\alpha^3\beta\gamma^2(\gamma^3+\alpha^2\beta)}{\gamma^3\alpha^2\beta-\alpha^3\beta\gamma^2}\\
&=\frac{\gamma(\alpha^3+\beta\gamma^2)-\alpha(\gamma^3+\alpha^2\beta)}{\gamma-\alpha}\\
&=-\gamma\alpha^2-\gamma^2\alpha+\beta\gamma^2+\beta\gamma\alpha+\beta\alpha^2,
\end{align*}
$R$の座標は
\begin{align*}
r
&=\frac{\alpha^3\beta^2\gamma(\beta^3+\omega^2\gamma\alpha^2)-\beta^3\omega^2\gamma\alpha^2(\alpha^3+\beta^2\gamma)}{\alpha^3\beta^2\gamma-\beta^3\omega^2\gamma\alpha^2}\\
&=\frac{\alpha(\beta^3+\omega^2\gamma\alpha^2)-\omega^2\beta(\alpha^3+\beta^2\gamma)}{\alpha-\omega^2\beta}\\
&=-\omega^2\alpha\beta^2-\omega\alpha^2\beta+\omega^2\gamma\alpha^2+\omega\gamma\alpha\beta+\gamma\beta^2
\end{align*}
と計算できる.
$p+\omega q+\omega^2r=0$であるため, 三角形$PQR$は正三角形である.
\end{ifsol*}
\begin{bprb}[Newtonの定理]
四角形$ABCD$が円$\Gamma$に外接している.
対角線$AC$の中点を$M$, 対角線$BD$の中点を$N$とする.
$M\neq N$であるとき, 円$\Gamma$の中心は直線$MN$上にあることを示せ.
\end{bprb}
\begin{ifsol*}
$\Gamma$が単位円となるような座標で考える.
円$\Gamma$と辺$AB$, $BC$, $CD$, $DA$との接点をそれぞれ$X$, $Y$, $Z$, $W$とおく.
このとき,
\[a=\frac{2wx}{w+x},\quad b=\frac{2xy}{x+y},\quad c=\frac{2yz}{y+z},\quad d=\frac{2zw}{z+w}\]
であり,
\[m=\frac{(y+z)wx+(w+x)yz}{(w+x)(y+z)},\quad n=\frac{(z+w)xy+(x+y)zw}{(x+y)(z+w)}\]
が得られる.
これにより,
\begin{align*}
\frac nm=\frac{(w+x)(y+z)}{(x+y)(z+w)}\sim\frac{\sqrt{wx}\sqrt{yz}}{\sqrt{xy}\sqrt{zw}}=1
\end{align*}
であり, $0$, $m$, $n$は共線である.
\end{ifsol*}
\begin{bprb}
三角形$ABC$の内心を$I$, 外心を$O$, Feuerbach点を$F_e$とする.
辺$BC$の中点を$M$, 三角形$ABC$の内接円が辺$BC$と接する点を$D$とするとき, 三角形$AIO$と三角形$F_eDM$とは相似であることを示せ.
\end{bprb}
\begin{ifsol*}
$i=0$, $\lvert d\rvert=1$となるような座標で考える.
定理\ref{thm:circumcenter-i}, 定理\ref{thm:feuerbach}により
\begin{align*}
a&=\frac{2ef}{e+f},\\
i&=0,\\
o&=\frac{2def(d+e+f)}{(d+e)(e+f)(f+d)},\\
f_e&=\frac{de+ef+fd}{d+e+f},\\
d&=d,\\
m&=\frac{d(de+df+2ef)}{(d+e)(d+f)}
\end{align*}
である.
示すべき式は
\[\begin{vmatrix}
\frac{2ef}{e+f}&\frac{de+ef+fd}{d+e+f}&1\\
0&d&1\\
\frac{2def(d+e+f)}{(d+e)(e+f)(f+d)}&\frac{d(de+df+2ef)}{(d+e)(d+f)}&1
\end{vmatrix}=0\]
である.

\begin{align*}
&\phantom{={}}\begin{vmatrix}
\frac{2ef}{e+f}&\frac{de+ef+fd}{d+e+f}&1\\
0&d&1\\
\frac{2def(d+e+f)}{(d+e)(e+f)(f+d)}&\frac{d(de+df+2ef)}{(d+e)(d+f)}&1
\end{vmatrix}\\
&=\frac{2ef}{(d+e)(e+f)(f+d)}\begin{vmatrix}
1&\frac{de+ef+fd}{d+e+f}&1\\
0&d&1\\
d(d+e+f)&d(de+df+2ef)&(d+e)(d+f)
\end{vmatrix}\\
&=\frac{2ef}{(d+e)(e+f)(f+d)(d+e+f)}\begin{vmatrix}
d+e+f&de+ef+fd&d+e+f\\
0&d&1\\
d(d+e+f)&d(de+df+2ef)&(d+e)(d+f)
\end{vmatrix}\\
&=\frac{2ef}{(d+e)(e+f)(f+d)(d+e+f)}\begin{vmatrix}
d+e+f&de+ef+fd&d+e+f\\
0&d&1\\
0&def&ef
\end{vmatrix}\\
&=0
\end{align*}
により示された.
\end{ifsol*}
\begin{bprb}
円周$\Gamma$上に$4$点$A$, $B$, $C$, $D$がある.
$A$, $B$, $C$, $D$における$\Gamma$の接線をそれぞれ$l_A$, $l_B$, $l_C$, $l_D$とする.
このとき, $l_A$と$l_B$との交点, $l_C$と$l_D$との交点, 直線$AC$と直線$BD$との交点は同一直線上にあることを示せ.
\end{bprb}
\begin{ifsol*}
示すべきことは$3$点$\frac{2ab}{a+b}$, $\frac{2cd}{c+d}$, $\frac{ac(b+d)-bd(a+c)}{ac-bd}$の共線であり, これらの共軛がそれぞれ$\frac{2}{a+b}$, $\frac{2}{c+d}$, $\frac{a+c-b-d}{ac-bd}$であることから示すべき式は
\[\begin{vmatrix}2ab&2&a+b\\2cd&2&c+d\\ac(b+d)-bd(a+c)&a+c-b-d&ac-bd\end{vmatrix}=0\]
である.
\begin{align*}
&\phantom{={}}\begin{gmatrix}[v]2ab&2&a+b\\2cd&2&c+d\\ac(b+d)-bd(a+c)&a+c-b-d&ac-bd\colops\add[ac]{1}{0}\add[-(a+c)]{2}{0}\end{gmatrix}\\
&=\begin{vmatrix}2ab+2ac-(a+c)(a+b)&2&a+b\\2cd+2ac-(a+c)(c+d)&2&c+d\\0&a+c-b-d&ac-bd\end{vmatrix}\\
&=(a-c)\begin{vmatrix}b-a&2&a+b\\c-d&2&c+d\\0&a+c-b-d&ac-bd\end{vmatrix}\\
&=(a-c)\Bigl(-(a+c-b-d)\bigl((b-a)(c+d)-(c-d)(a+b)\bigr)+(ac-bd)(2b-2a-2c+2d)\Bigr)\\
&=0
\end{align*}
と計算できるので$\frac{2ab}{a+b}$, $\frac{2cd}{c+d}$, $\frac{ac(b+d)-bd(a+c)}{ac-bd}$の共線が示された.
\end{ifsol*}
\begin{bprb}[Simsonの定理]
三角形$ABC$と点$P$とがある.
$P$から直線$BC$, $CA$, $AB$に下ろした垂線の足をそれぞれ$P_a$, $P_b$, $P_c$とする.
$A$, $B$, $C$, $P$が共円であることと$P_a$, $P_b$, $P_c$が共線であることとは同値であることを示せ.
\end{bprb}
\begin{ifsol*}
$\lvert a\rvert=\lvert b\rvert=\lvert c\rvert=1$となるような座標で考える.
\[p_a=\frac 12(b+c+p-bc\bar p),\quad p_b=\frac 12(c+a+p-ca\bar p),\quad p_c=\frac 12(a+b+p-ab\bar p)\]
であるので, $P_a$, $P_b$, $P_c$が共線であることは
\[\begin{vmatrix}
b+c+p-bc\bar p&\frac 1b+\frac 1c+\bar p-\frac p{bc}&1\\
c+a+p-ca\bar p&\frac 1c+\frac 1a+\bar p-\frac p{ca}&1\\
a+b+p-ab\bar p&\frac 1a+\frac 1b+\bar p-\frac p{ab}&1
\end{vmatrix}=0\]
と同値である.
ここで,
\begin{align*}
\begin{vmatrix}
b+c+p-bc\bar p&\frac 1b+\frac 1c+\bar p-\frac p{bc}&1\\
c+a+p-ca\bar p&\frac 1c+\frac 1a+\bar p-\frac p{ca}&1\\
a+b+p-ab\bar p&\frac 1a+\frac 1b+\bar p-\frac p{ab}&1
\end{vmatrix}
&=\begin{vmatrix}
b+c-bc\bar p&\frac 1b+\frac 1c-\frac p{bc}&1\\
c+a-ca\bar p&\frac 1c+\frac 1a-\frac p{ca}&1\\
a+b-ab\bar p&\frac 1a+\frac 1b-\frac p{ab}&1
\end{vmatrix}\\
&=\frac 1{abc}\begin{vmatrix}
b+c-bc\bar p&a(b+c-p)&1\\
c+a-ca\bar p&b(c+a-p)&1\\
a+b-ab\bar p&c(a+b-p)&1
\end{vmatrix}\\
&=\frac 1{abc}\begin{vmatrix}
b+c-bc\bar p&a(b+c-p)&1\\
(b-a)(c\bar p-1)&(b-a)(c-p)&0\\
(c-a)(b\bar p-1)&(c-a)(b-p)&0
\end{vmatrix}\\
&=\frac{(b-a)(c-a)}{abc}\begin{vmatrix}
c\bar p-1&c-p\\
b\bar p-1&b-p
\end{vmatrix}\\
&=\frac{(b-a)(c-a)(c-b)}{abc}(1-p\bar p)
\end{align*}
と計算できるので, $A$, $B$, $C$, $P$が共円であることと$P_a$, $P_b$, $P_c$が共線であることとは同値であることが示された.
\end{ifsol*}
\begin{bprb}[Pascalの定理]
単位円上に$6$点$A$, $B$, $C$, $D$, $E$, $F$があるとき, 直線$AB$と直線$DE$との交点, 直線$BC$と直線$EF$との交点, 直線$CD$と直線$FA$との交点は同一直線上にあることを示せ.
\end{bprb}
\begin{ifsol*}
直線$AB$と直線$DE$との交点, 直線$BC$と直線$EF$との交点, 直線$CD$と直線$FA$との交点の座標はそれぞれ
\[\frac{ab(d+e)-de(a+b)}{ab-de},\ \frac{bc(e+f)-ef(b+c)}{bc-ef},\ \frac{cd(f+a)-fa(c+d)}{cd-fa}\]
で与えられる.
$\overline{(\frac{ab(d+e)-de(a+b)}{ab-de})}=\frac{a+b-d-e}{ab-de}$などに注意すると,
\[\begin{vmatrix}
ab(d+e)-de(a+b)&a+b-d-e&ab-de\\
bc(e+f)-ef(b+c)&b+c-e-f&bc-ef\\
cd(f+a)-fa(c+d)&c+d-f-a&cd-fa
\end{vmatrix}
=0\]
を示せばよい.
\begin{align*}
&(c-f)\begin{pmatrix}ab(d+e)-de(a+b)\\ab-de\\a+b-d-e\end{pmatrix}
+(d-a)\begin{pmatrix}bc(e+f)-ef(b+c)\\bc-ef\\b+c-e-f\end{pmatrix}\\
&\phantom{{}={}}+(e-b)\begin{pmatrix}cd(f+a)-fa(c+d)\\cd-fa\\c+d-f-a\end{pmatrix}\\
&=\begin{pmatrix}0\\0\\0\end{pmatrix}
\end{align*}
であることを用いると, $c-f=d-a=e-b=0$のときとそうでないときとでどちらも
\[\begin{vmatrix}
ab(d+e)-de(a+b)&a+b-d-e&ab-de\\
bc(e+f)-ef(b+c)&b+c-e-f&bc-ef\\
cd(f+a)-fa(c+d)&c+d-f-a&cd-fa
\end{vmatrix}
=0\]
が成り立つことがわかる.
\end{ifsol*}
\begin{bprb}
三角形$ABC$と点$P$とがある.
$P$から直線$BC$, $CA$, $AB$に下ろした垂線の足をそれぞれ$P_a$, $P_b$, $P_c$とし, 三角形$ABC$に関する点$P$の等角共軛点を$Q$とする.
このとき, 三角形$P_aP_bP_c$の外心は線分$PQ$の中点であることを示せ.
\end{bprb}
\begin{ifsol*}
$\lvert a\rvert=\lvert b\rvert=\lvert c\rvert=1$となるような座標で考える.
\[p_a=\frac 12(b+c+p-bc\bar p),\quad p_b=\frac 12(c+a+p-ca\bar p),\quad p_c=\frac 12(a+b+p-ab\bar p)\]
が成り立つ.
また, 定理\ref{thm:isogonal_conjugate}により
\[q=\frac{abc\bar p^2-(ab+bc+ca)\bar p+a+b+c-p}{1-p\bar p}\]
である.
線分$PQ$の中点を$M$とする.
このとき,
\begin{align*}
m-p_a
&=\frac 12\bigl(p+q-(b+c+p-bc\bar p)\bigr)\\
&=\frac 1{2(1-p\bar p)}\bigl(abc\bar p^2-(ab+bc+ca)\bar p+a+b+c-p-(b+c-bc\bar p)(1-p\bar p)\bigr)\\
&=\frac 1{2(1-p\bar p)}(p-a)(b\bar p-1)(c\bar p-1)
\end{align*}
と計算できる.
これの絶対値は$a$, $b$, $c$について対称なので, 線分$PQ$の中点が三角形$P_aP_bP_c$の外心であることが示された.
\end{ifsol*}

\subsection{数学オリンピックの過去問}
\begin{bprb}[ISL2024-G1]
円に内接する四角形$ABCD$が$AC<BD<AD$および$\angle DBA<90^\circ$をみたしている.
$D$を通り$AB$に平行な直線上に点$E$をとったところ, $E$と$C$は直線$AD$に関して反対側にあり, $AC=DE$が成り立った.
$A$を通り$CD$に平行な直線上に点$F$をとったところ, $F$と$B$は直線$AD$に関して反対側にあり, $BD=AF$が成り立った.
$BC$の垂直二等分線と$EF$の垂直二等分線とは, 四角形$ABCD$の外接円上で交わることを示せ.
\end{bprb}
\begin{ifsol*}
$\lvert a\rvert=\lvert b\rvert=\lvert c\rvert=\lvert d\rvert=1$となるような座標で考える.
四角形$ABCD$の外接円を$\Omega$とおく.
$\sqrt{bc}$が点$A$を含まない弧$BC$上にあるように$\sqrt{bc}$の符号を定める.

$e-d$は$a-c$を$-\frac 12\angle BAC$だけ回転したものであるため, $e-d=(a-c)\sqrt{\frac bc}$が成り立つ.
同様に, $f-a$は$d-b$を$\frac 12\angle BDC$だけ回転したものであるため, $f-a=(d-b)\sqrt{\frac cb}$が成り立つ.

$-\sqrt{bc}$は$BC$の垂直二等分線上にあるので$-\sqrt{bc}$が$EF$の垂直二等分線上にあることを示せば十分である.
\[e+\sqrt{bc}=(a-c)\sqrt\frac bc+d+\sqrt{bc}=a\sqrt\frac bc+d\]
および
\[f+\sqrt{bc}=(d-b)\sqrt\frac cb+a+\sqrt{bc}=d\sqrt\frac cb+a\]
が成り立つので$\lvert e+\sqrt{bc}\rvert=\lvert f+\sqrt{bc}\rvert$であり, $-\sqrt{bc}$が$EF$の垂直二等分線上にあることが示された.
\end{ifsol*}
\begin{bprb}[IMO2024-4]
$AB<AC<BC$をみたす三角形$ABC$において, その内心と内接円をそれぞれ$I$, $\omega$とおく.
直線$BC$上の$C$と異なる点$X$を, $X$を通り直線$AC$に平行な直線が$\omega$に接するようにとる.
同様に, 直線$BC$上の$B$と異なる点$Y$を, $Y$を通り直線$AB$に平行な直線が$\omega$に接するようにとる.
直線$AI$と三角形$ABC$の外接円の交点のうち$A$でない方を$P$とする.
辺$AC$, $AB$の中点をそれぞれ$K$, $L$とおく.
このとき, $\angle KIL+\angle YPX=180^\circ$が成り立つことを示せ.
\end{bprb}
\begin{ifsol*}
$I$を中心とする座標で考える.
一般性を失わないため, $3$点$A$, $B$, $C$は反時計回りに並んでいると仮定する.
$\omega$と辺$BC$, $CA$, $AB$との接点をそれぞれ$D$, $E$, $F$とおく.

直線$AC$に平行な$\omega$の接線は, $e$におけるものと$-e$におけるものとがあるが, $X$が$C$と異なることから, $X$は$-e$における$\omega$の接線と直線$BC$との交点である.
これにより,
\[x=\frac{-2de}{d-e}=\frac{2ed}{e-d}\]
である.
同様に,
\[y=\frac{2fd}{f-d}\]
である.
$K$, $L$はそれぞれ辺$AC$, $AB$の中点なので
\[k=\frac{e(2df+de+ef)}{(d+e)(e+f)},\quad l=\frac{f(2de+df+ef)}{(d+f)(e+f)}\]
である.
また, $P$は$I$と$A$内の傍心$I_a$との中点なので定理\ref{thm:excenter-i}により
\[p=\frac{2def}{(d+e)(d+f)}\]
である.

$\lvert\triangle{IBC}\rvert<\frac 12\lvert\triangle{ABC}\rvert$なので$I$は直線$KL$に関して$A$と反対側にあり, $\angle KIL=\measuredangle KIL$である.
また, $B$, $X$, $D$, $Y$, $C$はこの順に並ぶので$\angle YPX=\measuredangle YPX$である.
したがって, $\angle KIL+\angle YPX=180^\circ$を示すには$\frac{l-i}{k-i}\frac{x-p}{y-p}\in\mathbb{R}$を示せばよい.
\begin{align*}
\frac{l-i}{k-i}\frac{x-p}{y-p}
&=\frac{\frac{f(2de+df+ef)}{(d+f)(e+f)}}{\frac{e(2df+de+ef)}{(d+e)(e+f)}}\cdot\frac{\frac{2ed}{e-d}-\frac{2def}{(d+e)(d+f)}}{\frac{2fd}{f-d}-\frac{2def}{(d+e)(d+f)}}\\
&=\frac{f(d+e)(2de+df+ef)}{e(d+f)(2df+de+ef)}\cdot\frac{e(f-d)(d+e+2f)}{f(e-d)(d+2e+f)}\\
&=\frac{(e+d)(f-d)}{(e-d)(f+d)}\cdot\frac{(2de+df+ef)(d+e+2f)}{(2df+de+ef)(d+2e+f)}\\
\end{align*}
であり, これは共軛をとっても不変であるため実数である.
したがって, $\angle KIL+\angle YPX=180^\circ$が示された.
\end{ifsol*}
\begin{bprb}[239 Open MO 2024 Grade8-9 p4]
三角形$ABC$があり, その内心を$I$とする.
$X$, $Y$はそれぞれ線分$BI$, $CI$の$I$側の延長線上の点で, $\angle IAX=\angle IBA$および$\angle IAY=\angle ICA$が成り立つ.
線分$IA$の中点, 線分$XY$の中点, 三角形$ABC$の外心の$3$点は同一直線上にあることを示せ.
\end{bprb}
\begin{ifsol*}
$\lvert a\rvert=\lvert b\rvert=\lvert c\rvert=1$となるような座標で考える.
$A$を含まない弧$BC$の中点, $B$を含まない弧$CA$の中点, $C$を含まない弧$AB$の中点をそれぞれ$M_A$, $M_B$, $M_C$とおき,
$m_a=-\sqrt{bc}$, $m_b=-\sqrt{ca}$, $m_c=-\sqrt{ab}$となるように$\sqrt a$, $\sqrt b$, $\sqrt c$の符号を定める.
線分$IA$の中点, 線分$XY$の中点, 三角形$ABC$の外心をそれぞれ$M_1$, $M_2$, $O$とおく.

$i=-\sqrt{ab}-\sqrt{bc}-\sqrt{ca}$により,
\[m_1=\frac{a-\sqrt{ab}-\sqrt{ac}-\sqrt{bc}}2\]
である.
\[\measuredangle M_BBA=\measuredangle IBA=\measuredangle IAX=\measuredangle M_AAX\]
により, $X$は$l(a,\sqrt{ab})$上にある.
これにより, $x$は$l(a,\sqrt{ab})$と$l(b,-\sqrt{ca})$との交点なので
\begin{align*}
x
&=\frac{a\sqrt{ab}(b-\sqrt{ca})+b\sqrt{ca}(a+\sqrt{ab})}{a\sqrt{ab}+b\sqrt{ca}}\\
&=\frac{ab+b\sqrt{ac}-a\sqrt{ac}+a\sqrt{bc}}{a+\sqrt{bc}}
\end{align*}
となる.
同様に,
\[y=\frac{ac+c\sqrt{ab}-a\sqrt{ac}+a\sqrt{bc}}{a+\sqrt{bc}}\]
となり,
\begin{align*}
m_2
&=\frac{x+y}2\\
&=\frac{a(b+c)+\sqrt{abc}(\sqrt b+\sqrt c)-a\sqrt a(\sqrt b+\sqrt c)+2a\sqrt{bc}}{2(a+\sqrt{bc})}\\
&=\frac{\sqrt a(\sqrt b+\sqrt c)\bigl(\sqrt a(\sqrt b+\sqrt c)+\sqrt bc-a\bigr)}{2(a+\sqrt{bc})}\\
&=-\frac{\sqrt a(\sqrt b+\sqrt c)}{(a+\sqrt{bc})}m_1\\
&\sim m_1
\end{align*}
が得られる.
したがって, $M_1$, $M_2$, $O$は共線である.
\end{ifsol*}
\begin{bprb}[Vietnum TST 2024-3]
鋭角不等辺三角形$ABC$において, $ABC$の内接円が辺$BC$, $CA$, $AB$と接する点をそれぞれ$D$, $E$, $F$とする.
$A$, $B$, $C$から対辺$BC$, $CA$, $AB$に下ろした垂線の足をそれぞれ$X$, $Y$, $Z$とする.
$EF$, $FD$, $DE$に関して$X$, $Y$, $Z$と対称な点をそれぞれ$A^\prime$, $B^\prime$, $C^\prime$とするとき, 三角形$ABC$と三角形$A^\prime B^\prime C^\prime$とは相似であることを示せ.
\end{bprb}
\begin{ifsol*}
$\lvert d\rvert=\lvert e\rvert=\lvert f\rvert=1$となるような座標で考える.
\[a=\frac{2ef}{e+f},\quad b=\frac{2fd}{f+d},\quad c=\frac{2de}{d+e}\]
である.
外心を中心とする座標では$x=\frac 12(a+b+c-bc/a)$と表されるので, 命題\ref{prop:translation-io}により
\[x=\frac{1}{(d+e)(e+f)(f+d)}(d^2e^2+e^2f^2+f^2d^2-d^4-2def(d+e+f))\]
である.
系\ref{cor:reflection2}により
\begin{align*}
a^\prime
&=e+f-ef\bar x\\
&=e+f-\frac{ef}{d^2(d+e)(e+f)(f+d)}(d^2e^2+d^2f^2+d^4-e^2f^2-2d^2(de+ef+fd))\\
&=\frac{1}{d^2(d+e)(e+f)(f+d)}\\
&\phantom{{}={}}\times\biggl((e+f)^2d^2(d^2+(e+f)d+ef)-ef\bigl(d^4-2(e+f)d^3+(e-f)^2d^2-e^2f^2\bigr)\biggr)\\
&=\frac{1}{d^2(d+e)(e+f)(f+d)}\bigl((e^2+ef+f^2)d^4+(e^3+e^2f+ef^2+f^3)d^3+e^3f^3\bigr)
\end{align*}
である.
対称性により,
\begin{align*}
b^\prime=\frac{1}{e^2(d+e)(e+f)(f+d)}\bigl((f^2+fd+d^2)e^4+(f^3+f^2d+fd^2+d^3)e^3+f^3d^3\bigr)
\end{align*}
および
\begin{align*}
c^\prime=\frac{1}{f^2(d+e)(e+f)(f+d)}\bigl((d^2+de+e^2)f^4+(d^3+d^2e+de^2+e^3)f^3+d^3e^3\bigr)
\end{align*}
も成り立つ.

以上により,
\begin{align*}
a^\prime-b^\prime
&=\frac{1}{d^2e^2(d+e)(e+f)(f+d)}\\
&\phantom{{}={}}\times\biggl(e^2\bigl((e^2+ef+f^2)d^4+(e^3+e^2f+ef^2+f^3)d^3+e^3f^3\bigr)\\
&\phantom{{}=\times\biggl(}-d^2\bigl((d^2+df+f^2)e^4+(d^3+d^2f+df^2+f^3)e^3+d^3f^3\bigr)\biggr)\\
&=\frac{d-e}{d^2e^2(d+e)(e+f)(f+d)}\Bigl(d^3e^3f+d^2e^2(d+e)f^2-d^3e^3(d+e)-d^3e^3f+d^2e^2f^3\\
&\phantom{=\frac{d-e}{d^2e^2(d+e)(e+f)(f+d)}\Bigl(}-(d^4+d^3e+d^2e^2+de^3+e^4)f^3\Bigr)\\
&=\frac{(d^2-e^2)(d^2e^2f^2-d^3e^3-e^3f^3-f^3d^3)}{d^2e^2(d+e)(e+f)(f+d)}
\end{align*}
および
\[
a-b=\frac{2ef}{e+f}-\frac{2df}{d+f}=\frac{2(e^2-d^2)f^2}{(d+e)(e+f)(f+d)}
\]
が成り立つ.
これにより,
\[\frac{a^\prime-b^\prime}{a-b}=\frac{d^3e^3+e^3f^3+f^3d^3-d^2e^2f^2}{2d^2e^2f^2}\]
であり, これは$d$, $e$, $f$について対称なので三角形$ABC$と三角形$A^\prime B^\prime C^\prime$とは相似である.
\end{ifsol*}
\begin{bprb}[Balkan MO 2025-2]
鋭角三角形$ABC$において, 垂心を$H$とし, 辺$BC$上(端点を除く)に点$D$をとる.
辺$AB$上(端点を除く)に点$E$を, 辺$AC$上(端点を除く)に点$F$をそれぞれとったところ, $4$点$A$, $C$, $D$, $E$および$4$点$A$, $B$, $D$, $F$はそれぞれ同一円周上にあった.
線分$BF$と線分$CE$との交点を$P$とおく.
直線$HA$上に点$L$を, 直線$LC$が三角形$PBC$の外接円と点$C$で接するようにとる.
直線$BH$と直線$CP$との交点を$X$とおく.
このとき, $3$点$D$, $X$, $L$は同一直線上にあることを示せ.
\end{bprb}
\begin{ifsol*}
$\lvert a\rvert=\lvert b\rvert=\lvert c\rvert=1$となるような座標で考える.
直線$AD$と三角形$ABC$の外接円との交点のうち$A$でない方を$Q$とする.
このとき,
\[d=\frac{aq(b+c)-bc(a+q)}{aq-bc}\]
である.
また,
\[\measuredangle ECB=\measuredangle ECD=\measuredangle EAD=\measuredangle BAQ\]
により, 直線$CE$は$l(c,\frac{b^2}q)$である.
$x$は$l(b,-\frac{ac}b)$と$l(c,\frac{b^2}q)$との交点なので
\begin{align*}
x
&=\frac{\frac{b^2c}q\bigl(b-\frac{ac}b\bigr)+ac\bigl(c+\frac{b^2}q\bigr)}{\frac{b^2c}q+ac}\\
&=\frac{b(b^2-ac)+a(b^2+cq)}{b^2+aq}\\
&=\frac{b^3-abc+ab^2+acq}{b^2+aq}
\end{align*}
である.
また,
\begin{align*}
\measuredangle LCB
&=\measuredangle CPB=-\measuredangle PBC-\measuredangle BCP\\
&=-\measuredangle FBD-\measuredangle DCE\\
&=-\measuredangle FAD-\measuredangle DAE\\
&=-\measuredangle FAE=-\measuredangle CAB=\measuredangle BAC
\end{align*}
により直線$CL$は$l(c,\frac{b^2}c)$なので
\begin{align*}
l
&=\frac{b^2\bigl(a-\frac{bc}a\bigr)+bc\bigl(c+\frac{b^2}c\bigr)}{b^2+bc}\\
&=\frac{b(a^2-bc)+a(c^2+b^2)}{a(b+c)}\\
&=\frac{a^2b-b^2c+ac^2+ab^2}{a(b+c)}
\end{align*}
である.

以上により, 示すべきことは$3$点$\frac{aq(b+c)-bc(a+q)}{aq-bc}$, $\frac{b^3-abc+ab^2+acq}{b^2+aq}$, $\frac{a^2b-b^2c+ac^2+ab^2}{a(b+c)}$の共線である.
これらの共軛がそれぞれ$\frac{a+q-b-c}{aq-bc}$, $\frac{acq-b^2q+bcq+b^3}{bc(b^2+aq)}$, $\frac{bc^2-a^2c+ab^2+ac^2}{abc(b+c)}$であることに注意すると
示すべき式は
\[\begin{vmatrix}
aq(b+c)-bc(a+q)&a+q-b-c&aq-bc\\
bc(b^3-abc+ab^2+acq)&acq-b^2q+bcq+b^3&bc(b^2+aq)\\
bc(a^2b-b^2c+ac^2+ab^2)&bc^2-a^2c+ab^2+ac^2&abc(b+c)
\end{vmatrix}=0\]
である.
\begin{align*}
&\phantom{={}}\begin{vmatrix}
aq(b+c)-bc(a+q)&a+q-b-c&aq-bc\\
bc(b^3-abc+ab^2+acq)&acq-b^2q+bcq+b^3&bc(b^2+aq)\\
bc(a^2b-b^2c+ac^2+ab^2)&bc^2-a^2c+ab^2+ac^2&abc(b+c)
\end{vmatrix}\\
&=bc\begin{vmatrix}
aq(b+c)-bc(a+q)&bc(a+q-b-c)&aq-bc\\
b^3-abc+ab^2+acq&acq-b^2q+bcq+b^3&b^2+aq\\
a^2b-b^2c+ac^2+ab^2&bc^2-a^2c+ab^2+ac^2&ab+ac
\end{vmatrix}
\end{align*}
であり, 第$1$列に第$2$列の$1$倍と第$3$列の$-(b+c)$倍とを足すと
\begin{align*}
&\phantom{={}}\begin{gmatrix}[v]
aq(b+c)-bc(a+q)&bc(a+q-b-c)&aq-bc\\
b^3-abc+ab^2+acq&acq-b^2q+bcq+b^3&b^2+aq\\
a^2b-b^2c+ac^2+ab^2&bc^2-a^2c+ab^2+ac^2&ab+ac
\colops
\add{1}{0}
\add[-(b+c)]{2}{0}
\end{gmatrix}\\
&=\begin{vmatrix}
0&bc(a+q-b-c)&aq-bc\\
(a+b)(b-c)(b-q)&acq-b^2q+bcq+b^3&b^2+aq\\
(a+b)(b-c)(a-c)&bc^2-a^2c+ab^2+ac^2&ab+ac
\end{vmatrix}\\
&=(a+b)(b-c)\begin{gmatrix}[v]
0&bc(a+q-b-c)&aq-bc\\
b-q&acq-b^2q+bcq+b^3&b^2+aq\\
a-c&bc^2-a^2c+ab^2+ac^2&ab+ac
\rowops
\add[-1]{0}{1}
\end{gmatrix}\\
&=(a+b)(b-c)\begin{gmatrix}[v]
0&bc(a+q-b-c)&aq-bc\\
b-q&acq-b^2q+b^3+b^2c+bc^2-abc&b^2+bc\\
a-c&bc^2-a^2c+ab^2+ac^2&ab+ac
\colops
\add[b^2-ac]{0}{1}
\end{gmatrix}\\
&=(a+b)(b-c)\begin{vmatrix}
0&bc(a+q-b-c)&aq-bc\\
b-q&b^2c+bc^2&b^2+bc\\
a-c&b^2c+bc^2&ab+ac
\end{vmatrix}\\
&=bc(b+c)(a+b)(b-c)\begin{vmatrix}
0&a+q-b-c&aq-bc\\
b-q&1&b\\
a-c&1&a
\end{vmatrix}
\end{align*}
と計算できる.
\begin{align*}
\begin{vmatrix}
0&a+q-b-c&aq-bc\\
b-q&1&b\\
a-c&1&a
\end{vmatrix}
&=-(a+q-b-c)\begin{vmatrix}b-q&b\\a-c&a\end{vmatrix}+(aq-bc)\begin{vmatrix}b-q&1\\a-c&1\end{vmatrix}\\
&=0
\end{align*}
なので,
\[\begin{vmatrix}
aq(b+c)-bc(a+q)&a+q-b-c&aq-bc\\
bc(b^3-abc+ab^2+acq)&acq-b^2q+bcq+b^3&bc(b^2+aq)\\
bc(a^2b-b^2c+ac^2+ab^2)&bc^2-a^2c+ab^2+ac^2&abc(b+c)
\end{vmatrix}=0\]
であり, $D$, $X$, $L$の共線が示された.
\end{ifsol*}
\begin{ifsoll*}
\[\begin{vmatrix}
aq(b+c)-bc(a+q)&a+q-b-c&aq-bc\\
bc(b^3-abc+ab^2+acq)&acq-b^2q+bcq+b^3&bc(b^2+aq)\\
bc(a^2b-b^2c+ac^2+ab^2)&bc^2-a^2c+ab^2+ac^2&abc(b+c)
\end{vmatrix}=0\]
を示すことに帰着されるところまでは同じである.
左辺は$q$の高々$2$次式なので$3$つの$q$の値について成立が確かめられれば任意の$q$について成立することがわかる.
$q=b,c,-\frac{bc}a$について成立することを示す.
$ABC$は鋭角三角形なので$b$, $c$, $-\frac{bc}a$はどの$2$つも相異なる.

$q=b$のとき, $X=L=B$なので$D$, $X$, $L$は共線であり, 行列式は$0$となる.
$q=c$のとき, $D=C$で, $X$と$L$はいずれも$l(c,\frac{b^2}c)$上にあるので$D$, $X$, $L$は共線であり, 行列式は$0$となる.
$q=-\frac{bc}a$のとき, $X$は$l(b,-\frac{ac}b)$と$l(c,-\frac{ab}c)$との交点なので$X=H$であることに注意すると, $D$, $X$, $L$はいずれも$l(a,-\frac{bc}a)$上にあるため, この場合も行列式は$0$である.

以上により, 任意の$q$について上の行列式が$0$となることが示された.
\end{ifsoll*}
\begin{bprb}[IMO2023-2]
$AB<AC$なる鋭角三角形$ABC$があり, その外接円を$\Omega$とする.
点$S$を, $\Omega$の$A$を含む弧$CB$の中点とする.
$A$を通り$BC$に垂直な直線が直線$BS$と点$D$で交わり, $\Omega$と$A$と異なる点$E$で交わる.
$D$を通り$BC$と平行な直線が直線$BE$と点$L$で交わる.
三角形$BDL$の外接円を$\omega$とおくと, $\omega$と$\Omega$が$B$と異なる点$P$で交わった.
このとき, 点$P$における$\omega$の接線と直線$BS$が, $\angle BAC$の二等分線上で交わることを示せ.
\end{bprb}
\begin{ifsol*}
$\lvert a\rvert=\lvert b\rvert=\lvert c\rvert=1$となるような座標で考える.
直線$PD$と$\Omega$との交点のうち$P$でない方を$F$とし, $P$を通る$\omega$の接線と$\Omega$との交点のうち$P$でない方を$Q$とする.
$s=\sqrt{bc}$となるように$\sqrt{bc}$の符号を定める.

$e=-\frac{bc}{a}$である.
\[\measuredangle BPF=\measuredangle BPD=\measuredangle BLD=\measuredangle ELD=\measuredangle EBC\]
なので$f=\frac{bc}{e}=-a$である.
また,
\[\measuredangle FPQ=\measuredangle DPQ=\measuredangle DBP=\measuredangle SBP\]
により$q=\frac{fp}{s}=-\frac{ap}{\sqrt{bc}}$である.
$3$直線$AE$, $BS$, $PF$が共点なので系\ref{cor:concurrency2}により
\[\begin{vmatrix}ae&a+e&1\\bs&b+s&1\\pf&p+f&1\end{vmatrix}=0\]
である.
\begin{align*}
\begin{vmatrix}ae&a+e&1\\bs&b+s&1\\pf&p+f&1\end{vmatrix}
&=\begin{vmatrix}-bc&a-\frac{bc}{a}&1\\b\sqrt{bc}&b+\sqrt{bc}&1\\-ap&p-a&1\end{vmatrix}\\
&=\begin{vmatrix}-bc&a-\frac{2bc}{a}&1\\b\sqrt{bc}&b+\sqrt{bc}+\frac{b\sqrt{bc}}{a}&1\\-ap&-a&1\end{vmatrix}\\
&=\begin{vmatrix}-bc&2a-\frac{2bc}{a}&1\\b\sqrt{bc}&b+\sqrt{bc}+\frac{b\sqrt{bc}}{a}+a&1\\-ap&0&1\end{vmatrix}\\
&=\frac{a+\sqrt{bc}}{a}\begin{vmatrix}-bc&2(a-\sqrt{bc})&1\\b\sqrt{bc}&a+b&1\\-ap&0&1\end{vmatrix}\\
\end{align*}
であり, $a+\sqrt{bc}\neq 0$なので
\[\begin{vmatrix}-bc&2(a-\sqrt{bc})&1\\b\sqrt{bc}&a+b&1\\-ap&0&1\end{vmatrix}=0\]
が成り立つ.

示すべきことは$l(a,-\sqrt{bc})$, $l(b,\sqrt{bc})$, $l(p,-\frac{ap}{\sqrt{bc}})$が共点であることなので
\[\begin{vmatrix}
-a\sqrt{bc}&a-\sqrt{bc}&1\\b\sqrt{bc}&b+\sqrt{bc}&1\\-\frac{ap^2}{\sqrt{bc}}&p-\frac{ap}{\sqrt{bc}}&1
\end{vmatrix}=0\]
を示せばよい.
\begin{align*}
\begin{vmatrix}-a\sqrt{bc}&a-\sqrt{bc}&1\\b\sqrt{bc}&b+\sqrt{bc}&1\\-\frac{ap^2}{\sqrt{bc}}&p-\frac{ap}{\sqrt{bc}}&1\end{vmatrix}
&=\begin{vmatrix}-a\sqrt{bc}&a-\sqrt{bc}&1\\b\sqrt{bc}&b+\sqrt{bc}&1\\-\frac{ap^2}{\sqrt{bc}}-ap&0&1+\frac{p}{\sqrt{bc}}\end{vmatrix}\\
&=\biggl(1+\frac{p}{\sqrt{bc}}\biggr)\begin{vmatrix}-a\sqrt{bc}&a-\sqrt{bc}&1\\b\sqrt{bc}&b+\sqrt{bc}&1\\-ap&0&1\end{vmatrix}\\
\end{align*}
であり, ここで
\[\begin{vmatrix}2(a-\sqrt{bc})&1\\a+b&1\end{vmatrix}=\begin{vmatrix}a-\sqrt{bc}&1\\b+\sqrt{bc}&1\end{vmatrix}\]
および
\[\begin{vmatrix}-bc&2(a-\sqrt{bc})\\b\sqrt{bc}&a+b\end{vmatrix}=\begin{vmatrix}-a\sqrt{bc}&a-\sqrt{bc}\\b\sqrt{bc}&b+\sqrt{bc}\end{vmatrix}\]
により
\[\begin{vmatrix}-bc&2(a-\sqrt{bc})&1\\b\sqrt{bc}&a+b&1\\-ap&0&1\end{vmatrix}=\begin{vmatrix}-a\sqrt{bc}&a-\sqrt{bc}&1\\b\sqrt{bc}&b+\sqrt{bc}&1\\-ap&0&1\end{vmatrix}\]
となるので,
\[\begin{vmatrix}
-a\sqrt{bc}&a-\sqrt{bc}&1\\b\sqrt{bc}&b+\sqrt{bc}&1\\-\frac{ap^2}{\sqrt{bc}}&p-\frac{ap}{\sqrt{bc}}&1
\end{vmatrix}=0\]
が示された.
\end{ifsol*}
\begin{bprb}[EGMO2025-3]\plabel{prb:egmo2025-3}
鋭角三角形$ABC$の辺$BC$上に点$D$, $E$を, $4$点$B$, $D$, $E$, $C$がこの順に並び, さらに$BD=DE=EC$をみたすようにとる.
線分$AD$, $AE$の中点をそれぞれ$M$, $N$とする.
三角形$ADE$が鋭角三角形であると仮定し, その垂心を$H$とする.
直線$BM$, $CN$上にそれぞれ点$P$, $Q$をとると, $D$, $H$, $M$, $P$は同一円周上にある相異なる$4$点であり, $E$, $H$, $N$, $Q$もまた同一円周上にある相異なる$4$点であった.
このとき, $4$点$P$, $Q$, $N$, $M$は同一円周上にあることを示せ.
\end{bprb}
\begin{ifsol*}
$\lvert a\rvert=\lvert d\rvert=\lvert e\rvert=1$となるような座標で考える.
直線$PM$と直線$QN$との交点を$X$とする.
$X$が三角形$DHM$の外接円と三角形$EHN$の外接円との根軸上にあれば, 方冪の定理により$4$点$P$, $Q$, $N$, $M$は共円である.
したがって, 直線$BM$, 直線$CN$, 三角形$DHM$の外接円と三角形$EHN$の外接円との根軸の共点を示すことに帰着された.

$b=2d-e$, $m=\frac{a+d}2$なので, 直線$BM$の方程式は
\[\biggl(\frac{2e-d}{de}-\frac{a+d}{2ad}\biggr)z-\biggl(2d-e-\frac{a+d}2\biggr)\bar z=\frac{2e-d}{de}\frac{a+d}2-(2d-e)\frac{a+d}{2ad}\]
すなわち
\[\frac{3ae-2ad-de}{2ade}-\frac{3d-2e-a}2\bar z=\frac{a+d}{2ade}(e^2+2ae-2de-ad)\]
であり, 直線$CN$の方程式は
\[\frac{3ad-2ae-de}{2ade}-\frac{3e-2d-a}2\bar z=\frac{a+e}{2ade}(d^2+2ad-2de-ae)\]
である.
$h=a+d+e$なので, 三角形$DHM$の外接円の方程式は
\[\begin{vmatrix}
z\bar z&z&\bar z&1\\
\frac{(a+d+e)(ad+de+ea)}{ade}&a+d+e&\frac{ad+de+ea}{ade}&1\\
\frac{(a+d)^2}{4ad}&\frac{a+d}2&\frac{a+d}{2ad}&1\\
1&d&\frac 1d&1
\end{vmatrix}=0\]
である.
ここで,
\begin{align*}
&\phantom{={}}\begin{vmatrix}
z\bar z&z&\bar z&1\\
\frac{(a+d+e)(ad+de+ea)}{ade}&a+d+e&\frac{ad+de+ea}{ade}&1\\
\frac{(a+d)^2}{4ad}&\frac{a+d}2&\frac{a+d}{2ad}&1\\
1&d&\frac 1d&1
\end{vmatrix}\\
&=\frac{1}{4a^2d^3e}\begin{gmatrix}[v]
z\bar z&z&\bar z&1\\
(a+d+e)(ad+de+ea)&ade(a+d+e)&ad+de+ea&ade\\
(a+d)^2&2ad(a+d)&2(a+d)&4ad\\
d&d^2&1&d
\rowops
\add[-ae]{3}{1}
\end{gmatrix}\\
&=\frac{a+e}{4a^2d^3e}\begin{gmatrix}[v]
z\bar z&z&\bar z&1\\
(a+d)(d+e)&ade&d&0\\
(a+d)^2&2ad(a+d)&2(a+d)&4ad\\
d&d^2&1&d
\rowops
\add[-(a+d)]{3}{2}
\end{gmatrix}\\
&=\frac{(a+e)(a-d)}{4a^2d^3e}\begin{vmatrix}
z\bar z&z&\bar z&1\\
(a+d)(d+e)&ade&d&0\\
a+d&2d(a+d)&0&2d\\
d&d^2&1&d
\end{vmatrix}
\end{align*}
であり,
\[\begin{vmatrix}
ade&d&0\\
2d(a+d)&0&2d\\
d^2&1&d
\end{vmatrix}=-2ad^2(d+e),\]
\[\begin{vmatrix}
(a+d)(d+e)&d&0\\
a+d&0&2d\\
d&1&d
\end{vmatrix}=-d(d^2+3ad+2de+2ae),\]
\[\begin{vmatrix}
(a+d)(d+e)&ade&0\\
a+d&2d(a+d)&2d\\
d&d^2&d
\end{vmatrix}=ad^2(2d^2+2ad+3de+ae),\]
\[\begin{vmatrix}
(a+d)(d+e)&ade&d\\
a+d&2d(a+d)&0\\
d&d^2&1
\end{vmatrix}=d(a+d)(d^2+ae+2ad+2de)\]
により, 三角形$DHM$の外接円の方程式は
\begin{align*}
-2ade(d+e)z\bar z+e(d^2+3ad+2de+2ae)z+ade(2d^2+2ad+3de+ae)\bar z\\
=e(a+d)(d^2+ae+2ad+2de)
\end{align*}
である.
同様に, 三角形$EHN$の外接円の方程式は
\begin{align*}
-2ade(d+e)z\bar z+d(e^2+3ae+2de+2ad)z+ade(2e^2+2ae+3de+ad)\bar z\\
=d(a+e)(e^2+ad+2ae+2de)
\end{align*}
である.
これら$2$式の差をとると, 三角形$DHM$の外接円と三角形$EHN$の外接円との根軸の方程式は
\[(2ad+2ae+de)z-ade(a+2d+2e)\bar z=(d+e)(a^2-de)\]
となる.

以上により, 示すべき式は
\[\begin{vmatrix}
2ad+2ae+de&-ade(a+2d+2e)&(d+e)(a^2-de)\\
3ae-2ad-de&-ade(3d-2e-a)&(a+d)(e^2+2ae-2de-ad)\\
3ad-2ae-de&-ade(3e-2d-a)&(a+e)(d^2+2ad-2de-ae)
\end{vmatrix}=0\]
すなわち
\[\begin{vmatrix}
2ad+2ae+de&a+2d+2e&(d+e)(a^2-de)\\
3ae-2ad-de&3d-2e-a&(a+d)(e^2+2ae-2de-ad)\\
3ad-2ae-de&3e-2d-a&(a+e)(d^2+2ad-2de-ae)
\end{vmatrix}=0\]
である.

\begin{align*}
&\phantom{={}}\begin{gmatrix}[v]
2ad+2ae+de&a+2d+2e&(d+e)(a^2-de)\\
3ae-2ad-de&3d-2e-a&(a+d)(e^2+2ae-2de-ad)\\
3ad-2ae-de&3e-2d-a&(a+e)(d^2+2ad-2de-ae)
\rowops
\add{0}{1}
\add{0}{2}
\end{gmatrix}\\
&=\begin{vmatrix}
2ad+2ae+de&a+2d+2e&(d+e)(a^2-de)\\
5ae&5d&-(a+3e)d^2+ae(3a+e)\\
5ad&5e&-(a+3d)e^2+ad(3a+d)
\end{vmatrix}\\
&=5\begin{gmatrix}[v]
2ad+2ae+de&a+2d+2e&5(d+e)(a^2-de)\\
ae&d&-(a+3e)d^2+ae(3a+e)\\
ad&e&-(a+3d)e^2+ad(3a+d)
\rowops
\add[-2]{1}{0}
\add[-2]{2}{0}
\end{gmatrix}\\
&=5\begin{gmatrix}[v]
de&a&-a^2(d+e)+de(d+e)\\
ae&d&-(a+3e)d^2+ae(3a+e)\\
ad&e&-(a+3d)e^2+ad(3a+d)
\colops
\add[ad+de+ea]{1}{2}
\add[-(a+d+e)]{0}{2}
\end{gmatrix}\\
&=5\begin{gmatrix}[v]
de&a&0\\
ae&d&-2d^2e+2a^2e\\
ad&e&-2de^2+2a^2d
\colops
\add[2de]{1}{2}
\add[-2a]{0}{2}
\end{gmatrix}\\
&=5\begin{vmatrix}
de&a&0\\
ae&d&0\\
ad&e&0
\end{vmatrix}\\
&=0
\end{align*}
により, $3$直線の共点が示された.
\end{ifsol*}
\begin{bprb}[JMO春合宿2025-6, ISL2024-G7]\plabel{prb:isl2024-g7}
$AB<AC<BC$をみたす三角形$ABC$の内心を$I$とし, 直線$AI$, $BI$, $CI$と三角形$ABC$の外接円の交点のうち, それぞれ$A$, $B$, $C$でない方を$M_A$, $M_B$, $M_C$とする.
直線$AI$と辺$BC$が点$D$で交わっており, 半直線$BM_C$と半直線$CM_B$が点$X$で交わっている.
また, 三角形$XBC$の外接円と三角形$XM_BM_C$の外接円が$X$でない点$S$で交わっており, 直線$BX$, $CX$と三角形$SM_AX$の外接円の交点のうち, $X$でない方をそれぞれ$P$, $Q$とする.
このとき, 三角形$DIS$の外心は直線$PQ$上にあることを示せ.
\end{bprb}
\begin{ifsol*}
$\lvert a\rvert=\lvert b\rvert=\lvert c\rvert=1$となるような座標で考える.
$m_a=-\sqrt{bc}$, $m_b=-\sqrt{ca}$, $m_c=-\sqrt{ab}$となるように$\sqrt a$, $\sqrt b$, $\sqrt c$の符号を定める.

$S$の定義により, $S$は四角形$BCM_BM_C$のMiquel点であり, $\triangle IBC\sim\triangle AM_CM_B$に注意すると, $4$点$(S,A,M_C,M_B)$と$(S,I,B,C)$とは相似である.
さらに, $\triangle SM_CM_B\sim\triangle SBC\sim\triangle SPQ$が成り立つ.

$\triangle BIC\sim\triangle PDQ$を示す.
$3$点$(M_C,B,P^\prime)$と$3$点$(A,I,D)$とが相似になるように点$P^\prime$をとる.
$P^\prime$が三角形$SM_AX$の外接円上にあることを示せば, $P^\prime=P$がわかり, $\triangle SPD\sim\triangle SBI$が示される.
同様の方法で$\triangle SQD\sim\triangle SCI$も示されるので, $\triangle BIC\sim\triangle PDQ$が導かれる.
したがって, $\triangle BIC\sim\triangle PDQ$を示すには$S$, $M_A$, $X$, $P^\prime$が共円であることを示せば十分である.
また, $\measuredangle XP^\prime S=\measuredangle M_CP^\prime S=\measuredangle ADS$なので, 示すべき式は
\[\measuredangle ADS=\measuredangle XM_AS\]
である.

$D$は$l(b,c)$と$l(a,m_a)$との交点なので
\[d=\frac{bc(a-\sqrt{bc})+a\sqrt{bc}(b+c)}{bc+a\sqrt{bc}}=\frac{ab+ac-bc+a\sqrt{bc}}{a+\sqrt{bc}}\]
である.
$S$は四角形$BCM_BM_C$のMiquel点なので
\[s=\frac{bm_b-cm_c}{b+m_b-c-m_c}=\frac{c\sqrt{ba}-b\sqrt{ca}}{b+\sqrt{ca}-c-\sqrt{ba}}=-\frac{\sqrt{abc}}{\sqrt{a}+\sqrt{b}+\sqrt{c}}\]
である.
$X$は$l(b,m_c)$と$l(c,m_b)$との交点なので
\[x=\frac{-b\sqrt{ba}(c-\sqrt{ca})+c\sqrt{ca}(b-\sqrt{ba})}{-b\sqrt{ba}+c\sqrt{ca}}=\frac{\sqrt{bc}(\sqrt{bc}-\sqrt{ba}-\sqrt{ca})}{b+\sqrt{bc}+c}\]
である.
以上により,
\[a-d\sim a+\sqrt{bc}\sim\sqrt{a\sqrt{bc}},\]
\begin{align*}
s-d
&=\frac{-\sqrt{abc}(a+\sqrt{bc})-(ab+ac-bc+a\sqrt{bc})(\sqrt{a}+\sqrt{b}+\sqrt{c})}{(\sqrt{a}+\sqrt{b}+\sqrt{c})(a+\sqrt{bc})}\\
% &=\frac{-(\sqrt{b}+\sqrt{c})^2a\sqrt{a}-(b+\sqrt{bc}+c)(\sqrt{b}+\sqrt{c})a+bc(\sqrt{b}+\sqrt{c})}{(\sqrt{a}+\sqrt{b}+\sqrt{c})(a+\sqrt{bc})}\\
% &=-\frac{(\sqrt{b}+\sqrt{c})\bigl((\sqrt{b}+\sqrt{c})a\sqrt{a}+(b+\sqrt{bc}+c)a-bc\bigr)}{(\sqrt{a}+\sqrt{b}+\sqrt{c})(a+\sqrt{bc})}\\
% &=-\frac{(\sqrt{b}+\sqrt{c})(\sqrt{a}+\sqrt{b})\bigl((\sqrt{b}+\sqrt{c})a+c\sqrt{a}-c\sqrt{b}\bigr)}{(\sqrt{a}+\sqrt{b}+\sqrt{c})(a+\sqrt{bc})}\\
&=-\frac{(\sqrt{b}+\sqrt{c})(\sqrt{a}+\sqrt{b})(\sqrt{a}+\sqrt{c})(\sqrt{ab}+\sqrt{ac}-\sqrt{bc})}{(\sqrt{a}+\sqrt{b}+\sqrt{c})(a+\sqrt{bc})}\\
&\sim\sqrt{\sqrt{bc}}\frac{\sqrt{ab}+\sqrt{ac}-\sqrt{bc}}{\sqrt{a}+\sqrt{b}+\sqrt{c}},
\end{align*}
\begin{align*}
x-m_a
&=\frac{\sqrt{bc}(\sqrt{bc}-\sqrt{ba}-\sqrt{ca})+\sqrt{bc}(b+\sqrt{bc}+c)}{b+\sqrt{bc}+c}\\
&=\frac{\sqrt{bc}(\sqrt{b}+\sqrt{c})(\sqrt{b}+\sqrt{c}-\sqrt{a})}{b+\sqrt{bc}+c}\\
&\sim\sqrt{\sqrt{bc}}(\sqrt{b}+\sqrt{c}-\sqrt{a}),
\end{align*}
\[s-m_a=\frac{-\sqrt{abc}+\sqrt{bc}(\sqrt{a}+\sqrt{b}+\sqrt{c})}{\sqrt{a}+\sqrt{b}+\sqrt{c}}=\frac{\sqrt{bc}(\sqrt{b}+\sqrt{c})}{\sqrt{a}+\sqrt{b}+\sqrt{c}}\sim\frac{\sqrt{bc}\sqrt{\sqrt{bc}}}{\sqrt{a}+\sqrt{b}+\sqrt{c}}\]
が成り立つので
\[\frac{a-d}{s-d}\frac{s-m_a}{x-m_a}\sim\frac{\sqrt{abc}}{(\sqrt{ab}+\sqrt{ac}-\sqrt{bc})(\sqrt{b}+\sqrt{c}-\sqrt{a})}\]
である.
右辺は共軛をとっても変化しないため実数である.
したがって,
\[\measuredangle ADS=\measuredangle XM_AS\]
が示された.

これにより, $S$, $M_A$, $X$, $P^\prime$は共円であり, $\triangle BIC\sim\triangle PDQ$である.

直線$PQ$に関して$D$と対称な点を$E$とする.
$A$と$I$とは直線$M_BM_C$に関して対称なことに注意すると, $(S,A,M_C,M_B,I)$と$(S,D,P,Q,E)$とは相似である.
\[\measuredangle DES=\measuredangle AIS=\measuredangle DIS\]
となるので$E$は三角形$DIS$の外接円上にあり, 三角形$DIS$の外心は線分$DE$の垂直二等分線, すなわち直線$PQ$上にある.
\end{ifsol*}
\begin{bprb}[ISL2024-G5]
三角形$ABC$とその内心$I$とがあり, 三角形$BIC$の外接円を$\Omega$とする.
$K$は辺$BC$上(端点を除く)の点で, $\angle BAK<\angle KAC$をみたす.
$\angle BKA$の二等分線は$\Omega$と$2$点$W$, $X$で交わり, $\angle CKA$の二等分線は$\Omega$と$2$点$Y$, $Z$で交わった.
ただし, $W$および$Y$は直線$BC$に関して$A$と同じ側にあるとする.
このとき, $\angle WAY=\angle ZAX$を示せ.
\end{bprb}
\begin{ifsol*}
$\lvert b\rvert=\lvert c\rvert=\lvert i\rvert=1$となるような座標で考える.
定理\ref{thm:a-m}により$a=\frac{bc+i^2}{b+c}$である.
$\angle WAY=\measuredangle WAY$および$\angle ZAX=\measuredangle ZAX$が成り立つので, 示すべきことは$(w-a)(x-a)\sim(y-a)(z-a)$である.

直線$AK$と$\Omega$との交点を$P$, $Q$とする.
$A$は直線$PQ$上にあるので$l(p,q)$, $l(i,-i)$, $l(b,i^2/c)$は共線である.
したがって, 系\ref{cor:concurrency2}により
\begin{align*}
0
&=\begin{vmatrix}
pq&p+q&1\\
-i^2&0&1\\
\frac{bi^2}c&b+\frac{i^2}c&1
\end{vmatrix}\\
&=\frac 1c\begin{vmatrix}
pq&p+q&1\\
-i^2&0&1\\
bi^2&bc+i^2&c
\end{vmatrix}\\
&=\frac 1c\begin{vmatrix}
pq+i^2&p+q&0\\
-i^2&0&1\\
(b+c)i^2&bc+i^2&0
\end{vmatrix}\\
&=\frac{(p+q)(b+c)i^2-(pq+i^2)(bc+i^2)}{c}
\end{align*}
である.
これにより,
\[p+q=\frac{(pq+i^2)(bc+i^2)}{(b+c)i^2}\]
である.

直線$WX$は$\angle BKA$の二等分線なので$(wx)^2=bcpq$である.
$wx=\sqrt{bcpq}$, $yz=-\sqrt{bcpq}$となるように$\sqrt{bcpq}$の符号を定める.
直線$BC$, $PQ$, $WX$, $YZ$は共点なので
\[\begin{vmatrix}wx&w+x&1\\pq&p+q&1\\bc&b+c&1\end{vmatrix}=\begin{vmatrix}yz&y+z&1\\pq&p+q&1\\bc&b+c&1\end{vmatrix}=0\]
が成り立つ.
これにより
\[w+x=\frac{wx(p+q-b-c)+pq(b+c)-bc(p+q)}{pq-bc}\]
が成り立つ.

以上をあわせると
\begin{align*}
(w-a)(x-a)
&=wx-(w+x)\frac{bc+i^2}{b+c}+\biggl(\frac{bc+i^2}{b+c}\biggr)^2\\
&=wx-\frac{bc+i^2}{b+c}\frac{wx(p+q-b-c)+pq(b+c)-bc(p+q)}{pq-bc}+\biggl(\frac{bc+i^2}{b+c}\biggr)^2\\
&=wx-\frac{bc+i^2}{b+c}\frac{wx\Bigl(\frac{(pq+i^2)(bc+i^2)}{(b+c)i^2}-b-c\Bigr)+pq(b+c)-bc\frac{(pq+i^2)(bc+i^2)}{(b+c)i^2}}{pq-bc}+\biggl(\frac{bc+i^2}{b+c}\biggr)^2\\
&=\frac 1{(b+c)^2i^2(pq-bc)}\biggl(\Bigl((b+c)^2i^2(pq-bc)-(bc+i^2)\bigl((pq+i^2)(bc+i^2)-(b+c)^2i^2\bigr)\Bigr)wx\\
&\qquad\qquad-(bc+i^2)\bigl(pq(b+c)^2i^2-bc(pq+i^2)(bc+i^2)\bigr)+(bc+i^2)^2i^2(pq-bc)\biggr)\\
&=\frac 1{(b+c)^2i^2(pq-bc)}\biggl(\Bigl(-i^6+(b^2+c^2-pq)i^4+(b^2pq+c^2pq-b^2c^2)i^2-b^2c^2pq\Bigr)wx\\
&\qquad\qquad+\Bigl(i^6-(b^2-bc+c^2)i^4-(b^2-bc+c^2)bci^2+b^3c^3\Bigr)pq\biggr)
\end{align*}
と計算できる.
\begin{align*}
-i^6+(b^2+c^2-pq)i^4+(b^2pq+c^2pq-b^2c^2)i^2-b^2c^2pq&\sim i^3bc\sqrt{pq},\\
i^6-(b^2-bc+c^2)i^4-(b^2-bc+c^2)bci^2+b^3c^3&\sim i^3bc\sqrt{bc}
\end{align*}
から
\[(w-a)(x-a)\sim\frac{i^3bcpq\sqrt{bc}}{(b+c)^2i^2(pq-bc)}\]
がわかる.
$(y-a)(z-a)$は$(w-a)(x-a)$において$\sqrt{bcpq}$を$-\sqrt{bcpq}$に変えたものなので
\[(y-a)(z-a)\sim\frac{i^3bcpq\sqrt{bc}}{(b+c)^2i^2(pq-bc)}\]
もわかる.
したがって,
\[(w-a)(x-a)\sim(y-a)(z-a)\]
が示された.
\end{ifsol*}
\begin{bprb}[ISL2024-G6]\plabel{prb:isl2024-g6}
$AB<AC$なる鋭角三角形$ABC$があり, その外接円を$\Gamma$とする.
$\Gamma$上に点$X$, $Y$をとると, 直線$XY$と直線$BC$とは$\angle BAC$の外角の二等分線上で交わった.
$X$, $Y$における$\Gamma$の接線の交点を$T$とすると, $T$は$BC$に関して$A$と同じ側にあり, 直線$TX$, $TY$は直線$BC$とそれぞれ$U$, $V$で交わった.
三角形$TUV$の角$T$内の傍心を$J$とするとき, $AJ$は$\angle BAC$を二等分することを示せ.
\end{bprb}
\begin{ifsol*}
$\lvert a\rvert=\lvert b\rvert=\lvert c\rvert=1$となるような座標で考える.
$\sqrt{bc}$が$A$を含む弧$BC$上にあるように$\sqrt{bc}$の符号を定める.
$\angle UTV$の二等分線と$\angle BAC$の二等分線との交点を$J^\prime$とし, $J^\prime$が三角形$TUV$の角$T$内の傍心であることを示す.
$J^\prime$が$\angle VUT$の外角の二等分線上にあることを示せばよく, $\angle VUT$の外角の二等分線が$l(x,-\sqrt{bc})$に平行であることから, 示すべき式は
\[u-j^\prime=\overline{u-j^\prime}\cdot\sqrt{bc}x\]
である.

$\angle UTV$の二等分線は$l(\sqrt{xy},-\sqrt{xy})$であることから,
\[j^\prime=\frac{xy(a-\sqrt{bc})}{xy-a\sqrt{bc}}\]
である.
また,
\[u=\frac{x^2(b+c)-2bcx}{x^2-bc}\]
である.
以上より,
\begin{align*}
u-j^\prime
&=\frac{x^2(b+c)-2bcx}{x^2-bc}-\frac{xy(a-\sqrt{bc})}{xy-a\sqrt{bc}}\\
&=\frac{(x^2(b+c)-2bcx)(xy-a\sqrt{bc})-xy(a-\sqrt{bc})(x^2-bc)}{(x^2-bc)(xy-a\sqrt{bc})}\\
&=\frac{(b+c-a+\sqrt{bc})x^3y-2bcx^2y-a\sqrt{bc}(b+c)x^2+bc(a-\sqrt{bc})xy+2abc\sqrt{bc}x}{(x^2-bc)(xy-a\sqrt{bc})}\\
\end{align*}
であり,
\begin{align*}
\overline{u-j^\prime}
&=\frac{(ab+ac-bc)\sqrt{bc}+abc-2a\sqrt{bc}x-(b+c)xy-(a-\sqrt{bc})x^2+2x^2y}{(x^2-bc)(xy-a\sqrt{bc})}
\end{align*}
となる.
したがって,
\begin{align*}
&\phantom{{}={}}u-j-\overline{u-j^\prime}\cdot\sqrt{bc}x\\
&=\frac 1{(x^2-bc)(xy-a\sqrt{bc})}\\
&\phantom{={}}\times\biggl((b+c-a-\sqrt{bc})x^3y+(a-\sqrt{bc})\sqrt{bc}x^3+(b+c-2\sqrt{bc})\sqrt{bc}x^2y\\
&\phantom{=\times\biggl(}-(b+c-2\sqrt{bc})a\sqrt{bc}x^2+(a-\sqrt{bc})bcxy+(a\sqrt{bc}+bc-ab-ac)bcx\biggr)\\
&=\frac x{(x^2-bc)(xy-a\sqrt{bc})}\\
&\phantom{={}}\times\biggl((b+c-a-\sqrt{bc})x^2y+(a-\sqrt{bc})\sqrt{bc}x^2+(b+c-2\sqrt{bc})\sqrt{bc}xy\\
&\phantom{=\times\biggl(}-(b+c-2\sqrt{bc})a\sqrt{bc}x+(a-\sqrt{bc})bcy+(a\sqrt{bc}+bc-ab-ac)bc\biggr)
\end{align*}
となる.
以下,
\begin{align*}
&\phantom{={}}(b+c-a-\sqrt{bc})x^2y+(a-\sqrt{bc})\sqrt{bc}x^2+(b+c-2\sqrt{bc})\sqrt{bc}xy\\
&\phantom{={}}-(b+c-2\sqrt{bc})a\sqrt{bc}x+(a-\sqrt{bc})bcy+(a\sqrt{bc}+bc-ab-ac)bc\\
&=0
\end{align*}
を示す.

$l(x,y)$, $l(b,c)$, $l(a,\sqrt{bc})$が共点なので
\[\begin{vmatrix}
xy&x+y&1\\
bc&b+c&1\\
a\sqrt{bc}&a+\sqrt{bc}&1
\end{vmatrix}=0\]
であり, これから
\[(b+c-a-\sqrt{bc})xy=(bc-a\sqrt{bc})(x+y)+a\sqrt{bc}(b+c)-bc(a+\sqrt{bc})\]
が得られる.
これを用いると,
\begin{align*}
&\phantom{={}}(b+c-a-\sqrt{bc})x^2y+(a-\sqrt{bc})\sqrt{bc}x^2+(b+c-2\sqrt{bc})\sqrt{bc}xy\\
&\phantom{={}}-(b+c-2\sqrt{bc})a\sqrt{bc}x+(a-\sqrt{bc})bcy+(a\sqrt{bc}+bc-ab-ac)bc\\
&=(bc-a\sqrt{bc})(x^2+xy)+\bigl(a\sqrt{bc}(b+c)-bc(a+\sqrt{bc})\bigr)x\\
&\phantom{={}}+(a-\sqrt{bc})\sqrt{bc}x^2+(b+c-2\sqrt{bc})\sqrt{bc}xy\\
&\phantom{={}}-(b+c-2\sqrt{bc})a\sqrt{bc}x+(a-\sqrt{bc})bcy+(a\sqrt{bc}+bc-ab-ac)bc\\
&=(b+c-a-\sqrt{bc})\sqrt{bc}xy+(a-\sqrt{bc})bcx+(a-\sqrt{bc})bcy+(a\sqrt{bc}+bc-ab-ac)bc\\
&=0
\end{align*}
となる.
したがって,
\[u-j^\prime=\overline{u-j^\prime}\cdot\sqrt{bc}x\]
が示された.
\end{ifsol*}
