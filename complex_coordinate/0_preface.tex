本書は, 筆者が数学オリンピックの現役時および引退後に複素座標について勉強したことをまとめたものである.

複素座標とは, 平面上の点に複素数の座標を対応させることで平面幾何の問題を解く手法のことである.
$1$つの点に$2$つの実数を割りあてる直交座標や, $3$つの実数を割りあてる重心座標とは違い, 複素座標では$1$つの点に複素数$1$つのみが割りあてられるという特徴がある.
複素座標を初めて学んだときはその簡潔さと威力とに非常に感銘を受けたが, さらに勉強を進めると, その力が思っていた以上に強いことや複素座標そのものが興味深い考察の対象であることがわかった.
本書を通じて複素座標の強力さや面白さを伝えることができれば幸いである.

本書の構成にうつる.
第1章では, 複素座標の計算の礎となる行列式と偏角とについて解説した.
第2章では共線の条件などの基本的な命題を扱い, 第3章では第2章の命題を用いて具体的な点の座標を計算する.
また, 第4章には練習問題を多数掲載し, それらすべての解答を第6章に記した.

第2章までは初等幾何の知識はできるだけ使わないように努めたが, 第3章以降では初等幾何の知識を十分に使用する.
『数学オリンピック幾何への挑戦 ユークリッド幾何学をめぐる船旅』(通称: 船旅)やその原著の『Euclidean Geometry in Mathematical Olympiads』の第3章までの知識があれば読むのにはほとんど困らない.
本書には長い計算が多く登場するが, 計算過程をすべて書くと煩雑になることから過程を省略している箇所が多くある.
そのため, 紙とペンを脇に置いて読むことを推奨する.

最後に, 本書が多くの数学オリンピック受験生の手に渡ると幸いである.
%
\newpage
%定義集
\begin{itembox}[l]{\textbf{注意書き・定義集}}
\begin{itemize}
\item 座標はすべて複素座標で考える.
\item 虚数単位には$i$ではなく$\sqrt{-1}$を用いる.
\item 実数全体の集合を$\mathbb{R}$で, 複素数全体の集合を$\mathbb{C}$で表す.
\item 特に断らないかぎり点の座標を対応する小文字で表す.
\item $z$の複素共軛を$\bar{z}$や$\overline{z}$で表す.
\item $0$でない複素数の偏角を$\arg z$で表す.
\item $0$でない複素数$\alpha$, $\beta$に対して$\frac\beta\alpha\in\mathbb{R}$であることを$\alpha\sim\beta$と表す.
\item $2$点$a$, $b$を通る直線を$l(a,b)$で表す.
\item 線分$XY$の長さを$\lvert XY\rvert$や$XY$で, 三角形$XYZ$の面積を$\lvert\triangle XYZ\rvert$で表す.
\item 三角形$ABC$と三角形$DEF$とが正の向きに相似であるとは, 回転, 平行移動, 拡大縮小のみを用いて一方を他方に移すことができることをいう.
三角形$ABC$と三角形$DEF$とが負の向きに相似であるとは, 三角形$ABC$の鏡映が三角形$DEF$と正の向きに相似であることをいう.
\item $e^{\sqrt{-1}x}$は$\cos x+\sqrt{-1}\sin x$の略記である. $\lvert e^{\sqrt{-1}x}\rvert=1$や$e^{\sqrt{-1}(x+y)}=e^{\sqrt{-1}x}e^{\sqrt{-1}y}$が成り立つことを確認せよ.
勿論, $e^z$に$z=\sqrt{-1}x$を代入したものとして読んでも矛盾は生じない.
\end{itemize}
\end{itembox}

%人名かな対応表
本書に登場する人名と, よく使われるかな表記との対応を記す.
\begin{center}
\begin{tabular}{c|c}
人名&かな表記\\
\hline
Cram\'er&クラメル, クラメール\\
de Moivre&ド・モアブル, ド・モワブル\\
Feuerbach&フォイエルバッハ\\
Gergonne&ジュルゴンヌ, ジェルゴンヌ\\
Lemoine&ルモワーヌ\\
Miquel&ミケル\\
Morley&モーリー\\
Napoleon&ナポレオン\\
Newton&ニュートン\\
Pascal&パスカル\\
Simson&シムソン\\
Vandelmonde&ヴァンデルモンド
\end{tabular}
\end{center}